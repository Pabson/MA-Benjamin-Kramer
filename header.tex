\documentclass[pdftex,a4paper,12pt,twoside,ngerman,openright,abstracton]{scrreprt} % Verwendung von KOMA-Script

\usepackage{lmodern}			% Schriftart
%\usepackage{textcomp}			% weitere Symbole
\usepackage[T1]{fontenc}		% Umlaute in PDF, aber Probleme bei ß
%\usepackage[latin1]{inputenc}		% LaTeX-Dateien sind in latin1 codiert
\usepackage[utf8]{inputenc}		% LaTeX-Dateien sind in UTF-8 codiert

%\usepackage[ngerman]{babel}		% neue deutsche Trennung

\usepackage[pdftex]{graphicx}		% PNG, JPG, PDF Bilder
%\usepackage{verbatim}			% Ausgabe von LaTeX-Quelltext
\usepackage{listings}			% Paket fuer Quellcode-Listings
\lstset{language=C++,
        showstringspaces=false,
        frame=L,
        tabsize=2,
        morekeywords={aspect,advice,slice,pointcut},
        aboveskip=15pt,
        belowskip=15pt}
\usepackage{url}			% Links mit \url{http://...}
\usepackage[clearempty]{titlesec}	% erweiterte Kontrolle ueber Ueberschriftformatierung
\usepackage{booktabs}			% fuer brauchbare Tabellen
%\usepackage{multirow}			% fuer Tabellenfelder, die sich ueber mehrere Zeilen/Spalten erstrecken
\usepackage{fancyhdr}			% erweiterte Kontrolle ueber Kopf-/Fusszeilen

\usepackage{pgfplots}
% Deaktiviert für TeXlive 2012
%\pgfplotsset{compat=1.8}
\usepackage{pgfplotstable}
\usetikzlibrary{arrows,patterns,shapes,trees}
\usepackage{microtype}
\usepackage[all]{nowidow}
\usepackage{attrib}

\titleformat{\paragraph}[hang]{\normalfont\bfseries}{}{0pt}{}
\setcounter{secnumdepth}{4}

\pagestyle{fancy}
\fancyhf{}
\fancypagestyle{plain}{%
\fancyhf{}
%\fancyhead[L]{\slshape{\leftmark}}
%\fancyhead[R]{\slshape{\thesection}}
\fancyfoot[L]{}
\fancyfoot[R]{\thepage}
\renewcommand{\headrulewidth}{0.0pt}
}
\fancyhead[EL]{\slshape{\leftmark}}
\fancyhead[ER]{\slshape{\thesection}}
\fancyhead[OL]{\slshape{\thesection}}
\fancyhead[OR]{\slshape{\leftmark}}
\fancyfoot[EL]{\thepage}
\fancyfoot[ER]{}
\fancyfoot[OL]{}
\fancyfoot[OR]{\thepage}
\renewcommand{\headrulewidth}{0.4pt}
\renewcommand{\footrulewidth}{0.4pt}
\renewcommand{\chaptermark}[1]{\markboth{#1}{}}
\renewcommand{\sectionmark}[1]{\markright{\MakeUppercase{#1}}}

%\hyphenation{}

\setlength{\headheight}{15mm}	% legt Höhe des Leerraumes fest, der am oberen Seitenrand für Kopfzeile reserviert wird 
\setlength{\headsep}{10mm}	% legt den Abstand zwischen der Kopfzeile und dem Rumpf der Seite fest

\usepackage[
        pdftex,
%        a4paper,                      % Option is no longer used
        pdfauthor={Benjamin Kramer},
        pdftitle={Efficient Scans on Modern Hardware}
        pdfsubject={Master's Thesis},
        pdfkeywords={Scan, SIMD, BitWeaving},
        breaklinks,
        colorlinks,
        linkcolor=black,
        urlcolor=blue,
        citecolor=black
]{hyperref}

% Abkuerzungen richtig formatieren
\usepackage{xspace}
\newcommand{\vgl}{vgl.\@\xspace}
\newcommand{\zB}{z.\nolinebreak[4]\,\nolinebreak[4]B.\@\xspace}
\newcommand{\bzw}{bzw.\@\xspace}
\newcommand{\dahe}{d.\nolinebreak[4]\,h.\nolinebreak[4]\@\xspace}
\newcommand{\etc}{etc.\@\xspace}
\newcommand{\evtl}{evtl.\@\xspace}
\newcommand{\ggf}{ggf.\@\xspace}
\newcommand{\bzgl}{bzgl.\@\xspace}
\newcommand{\so}{s.\nolinebreak[4]\,\nolinebreak[4]o.\@\xspace}
\newcommand{\iA}{i.\nolinebreak[4]\,\nolinebreak[4]A.\@\xspace}
\newcommand{\sa}{s.\nolinebreak[4]\,\nolinebreak[4]a.\@\xspace}
\newcommand{\su}{s.\nolinebreak[4]\,\nolinebreak[4]u.\@\xspace}
\newcommand{\ua}{u.\nolinebreak[4]\,\nolinebreak[4]a.\@\xspace}
\newcommand{\og}{o.\nolinebreak[4]\,\nolinebreak[4]g.\@\xspace}
\newcommand{\oBdA}{o.\nolinebreak[4]\,\nolinebreak[4]B.\nolinebreak[4]\,d.\nolinebreak[4]\,A.\@\xspace}
\newcommand{\OBdA}{O.\nolinebreak[4]\,\nolinebreak[4]B.\nolinebreak[4]\,d.\nolinebreak[4]\,A.\@\xspace}
\newcommand{\bspw}{bspw.\@\xspace}
\newcommand{\zT}{z.\nolinebreak[4]\,\nolinebreak[4]T.\@\xspace}
\newcommand{\ca}{ca.\@\xspace}

%\setlength{\parindent}{0mm}		% Einrückung 1. Zeile eines Absatzes

%\makeindex
