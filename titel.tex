\thispagestyle{empty}
\vspace*{-2.5cm}
\newlength{\links}
\setlength{\links}{0cm}
\sf
%\LARGE

\hspace*{\links}
\begin{minipage}{12.5cm}
\includegraphics[width=7.75cm]{images/tud_logo_rgb}
%\hspace*{-0.25cm} \textbf{TECHNISCHE UNIVERSITÄT DORTMUND}\\
%\hspace*{-1.2cm} \rule{5mm}{5mm} \hspace*{0.1cm} FACHBEREICH INFORMATIK\\
\end{minipage}

\vspace*{3.75cm}

\hspace*{\links}
\hspace*{0.5cm}
\begin{minipage}{9cm}
\large
\begin{center}
{\LARGE Master's Thesis}\\
\vspace*{1cm}
% HIER den Titel der Arbeit eintragen
% Falls der Titel zu lang ist oder die einzelnen Wörter zu lang sind, kann unter Umständen die Titelseite nicht richtig erzeugt werden.
% Abhilfe schafft entweder die Veränderung der Schriftgröße: \LARGE -> \large -> \normalsize
% und/oder die Verringerung des Abstandes zum nächsten Textblock (siehe unten)
\bf{\LARGE Efficient Scans on Modern Hardware}\\
\vspace*{1.5cm}
\large Benjamin Kramer\\
September 21, 2015
\end{center}
\end{minipage}

% Hier kann man den Abstand zum nächsten Textblock justieren.
% Je nach Länge des Titels muss man bis auf 3 cm runtergehen.
\vspace*{4cm}

\hspace*{\links}

%\vspace*{1.5cm}

\vspace*{.6cm}

\hspace*{\links}
\begin{minipage}[b]{6cm}
\normalsize
\normalfont
\raggedright
Adviser: \\
Prof.\ Dr.\ rer.\ nat.\ Jens Teubner\\
Dr.-Ing.\ Christian Mathis\\
\end{minipage}

\definecolor{TUGreen}{rgb}{0.517,0.721,0.094}
\vspace*{2.5cm}
\hspace*{\links}
\begin{minipage}[b]{8cm}
\normalsize
\normalfont
\raggedright
Technische Universität Dortmund\\
Fakultät für Informatik\\
Lehrstuhl für Datenbanken und Informationssysteme
\end{minipage}
\begin{minipage}{8cm}
\hspace{2cm}\vspace{2cm}
\includegraphics[width=3cm]{images/SAP_2011_Logo}
\end{minipage}
