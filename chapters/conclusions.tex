\chapter{Conclusions and Future Work}

In this thesis multiple ways of performing full table scans in a column
store were analyzed and evaluated. The baseline is \simdscan{}, which works on
tightly packed data and, as the name says, makes use of SIMD. The key
observation was that \simdscan{} is only bound by memory bandwidth on a single
core on a current Intel Haswell server and by making full use of AVX~2. This
made \simdscan{} the baseline algorithm and possible improvements were looked
at. To qualify the new scan algorithm has to have a very tight inner loop as
the available memory bandwidth only allows for a small amount of instructions
to be executed per word loaded from memory. This also makes the use of wider
SIMD instructions very important as it reduces the number of instructions
executed per byte read from memory. After many options of possible compression
techniques were examined \bwv{} and \bs{} were chosen as they use an extremely
simple compression scheme which uses exactly the same amount of bits as
\simdscan{}. The same data size made comparing the different approaches
feasible.

The huge advantage of \bwv{} and \bs{} over \simdscan{} is the ability to abort
processing of an element before all bits of that element are loaded. Combined
with a memory layout that lets the processor avoid loading unnecessary bits
this can yield a significant reduction in memory bandwidth used. \bwv{} uses a
vertical bit layout where bits of different code words are stored together
depending on their position in the input word. \bs{} is very similar but uses a
vertical byte layout, where individual bytes keep their inner layout but are
placed into different memory locations. Nevertheless, early exiting adds a
probabilistic component to the analysis of the runtime of the scan, depending on
the input data, which makes the scan less predictable.

While analyzing the behavior of \bwv{} another extremely important observation
was made. Since a modern processor always loads data from RAM in constant-sized
cache lines only skipping a cache line in its entirety can avoid loading data
from main memory. In theory, the probability of early exiting depends on the
number of elements processed at the same time, and thus the register size. A
single element can trigger the worst case of loading all bits and the
probability that such an element is in the register grows with register size.
However, since early exiting only avoids memory traffic, the probability is
always fixed at the cache line size regardless of register size. This of course
only applies as long as the algorithm is only limited by memory bandwidth, and
not by other CPU resources such as instruction throughput.

In benchmarks the early pruning algorithms can easily beat \simdscan{} on
\textsc{Between} predicates with bit widths larger than 20 bits per word and
uniform random data. For small bit widths up to about 10 bits there is no
difference in scan runtime and in between the pruning scans are actually slower
than \simdscan{}. This is due to the branching overhead incurred by pruning
early and makes it much harder to make a clear recommendation.

While inequality predicates such as less than and greater than show little to
none sensitivity on the selectivity, equality predicate performance is directly
bound to the selectivity. The selectivities where early pruning becomes
significant for those predicates lie at $0.2\,\%$ for \bwv{} and $1.6\,\%$ for
\bs{} which is very low. For this use case pruning algorithms have very little
advantage over \simdscan{}.

In early pruning also lies the major difference in scan performance between
\bwv{} and \bs{}. \bs{} stores fewer elements per cache line so the early
pruning probability is higher. On the other hand, it also executes more
branches per word read from memory so the branching overhead is higher. In this
thesis \bs{} could not beat \bwv{} in terms of scan performance. The effect is
more pronounced for complex predicates such as \textsc{Between} than for
simpler predicates such as equivalence, mostly due to AVX~2 missing important
instructions for the operations performed by \bs{} increasing the total number
of instructions executed per word.

Another extremely important thing to keep in mind when choosing a scan
algorithm is the time needed to bring data into the representation needed for
the scan and also the performance of the reverse operation which is needed to
materialize the results. Testing shows that \bwv{} is prohibitively expensive
in this regard, transforming the data takes orders of magnitudes longer than a
scan for this format. This is due to the gathering and scattering of bits over
many different words, AVX~2 provides no fast way to do this. For \simdscan{}
and \bs{} though, the packing and unpacking operations are very fast and close
to a simple copying operation. This makes it feasible to use the scan
representation as the only representation and going without additional indexes.

The last improvement evaluated is block-wise scanning, where multiple columns
are scanned and the switching between the columns occurs after a specific block
size. This reduces the amount of intermediate results that need to be written
to memory. The gain from this is not very large and can trigger effects in the
processor that can massively slow down the scan as the processor fails to
recognize the memory access pattern as a sequential scan.

The result of the benchmarks is that small block sizes (less than a couple of
thousand elements per block) perform a few percent better than a full column
scan for less than eight columns. For more columns the processor fails to
recognize the scan and performance becomes extremely bad. This effect goes away
for larger block sizes as the tested block size of $2^{17}$. The performance
advantage stays though. All in all block-wise scanning is only recommended if
it can be easily implemented in the data base system as the gains are rather
small.

In conclusion, \bwv{} and \bs{} are viable improvements over \simdscan{} in
some cases. The main advantage is visible for range predicates and large bit
widths over 20 bits. For all other cases results are more mixed and can be the
same or significantly worse. If a database system wants to make use of those
algorithms it should only be a secondary option and be picked based on the work
load.

\section{Future Work}

The most immediate open question is whether \bs{} can be made as fast as \bwv{}
on current or future hardware. The vertical byte format is vastly superior to
the vertical bit format in terms of packing and unpacking performance.

On the scan side, evaluating more compression formats is a very wide topic.
There are many different compression formats~\cite{lemire2015decoding} that can
potentially run fast enough to be only limited by memory bandwidth. There are
also formats such as Simple-8~\cite{Simple8} that may benefit from SIMD
instructions but actually implementing it on top of existing instruction sets
like AVX~2 is difficult. Another possibility is changing the order of values in
the column~\cite{markl1999variable}. This has the disadvantage of requiring meta
data to reconstruct the original order but can facilitate better compression.

Another potentially large field for future work is how to deal with different
scan algorithms in a query optimizer. This ranges from reordering the columns in
a query by expected selectivity based on historical data to increase early
pruning probabilities to the exact point when to choose a format like \bwv{}
over \simdscan{} and vice-versa. A fast \bs{} would make this a little simpler
as the fast packing and unpacking makes switching between representations
feasible.

If a vertical format is chosen as a primary representation there are also
possibilities to perform aggregate operations like summation directly on the
data. The same applies to joins. This would remove the need for unpacking
operations for queries that do not actually materialize results from the table.

Last but not least this thesis only looked at single-core performance. To use
the memory bandwidth to its full potential multiple cores are needed as a
single core cannot saturate the bus entirely. For example the Haswell CPU used
in the benchmarks peaks at around 11 gigabytes per second on a single core,
while the specification suggests a peak bandwidth of 68 gigabytes per second so
the processor is still not at its full potential.
