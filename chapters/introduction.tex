\chapter{Introduction}

The scan is one of the most basic primitives for building a database system. It
searches over a column in the database and checks for every entry in the column
whether it satisfies a predicate. The result is usually stored in a bit vector
of the same size as the column, indicating the status of the predicate for
every value, or a list of indexes.

A scan is not a computationally intensive task, but involves a lot of data so
its performance is directly related to how fast the data can be loaded.
Historically this meant that the scan runtime was equivalent to the speed of
the available non-volatile memory, typically hard drives with magnetic rotating
disks. In more recent time the storage for the data has shifted from hard
drives to main memory as it became cheaper. Main memory is orders of magnitudes
faster than hard drives, so the processor became the bottleneck. However,
processors also became better -- at a much faster pace than main memory --
so the processor gave its bottleneck position to main memory.

This thesis focuses on that bottleneck with respect to fast column cans and
explores ways to reduce the required bandwidth from main memory without shifting
the bottleneck back to the processor.

\section{Motivation and Goals}

The main motivation of this thesis is to explore and evaluate the use of modern
hardware technologies to accelerate scans. It builds upon existing research
results to reduce memory bandwidth such as \emph{BitWeaving}~\cite{BitWeaving}
and \simdscan{}~\cite{SIMD-Scan} and evaluate their effect on modern hardware
platforms and for different workloads.

The goal is to develop strategies to optimize the scan operation to leverage
the full capabilities of modern hardware. This includes identifying the
critical bottlenecks of existing scanning approaches and coming up with
strategies to reduce the load at those critical points.

\section{Thesis Structure}

Following this introduction chapter, Chapter~\ref{chapter:background} discusses
the theoretical background of the problems faced with fast scan algorithm and
introduces several improved scan approaches. Chapter~\ref{chapter:analysis}
tries to quantify the improvements and shows in which cases they work and in
which cases they will not. Details on the implementation of the scan algorithms
are laid down in Chapter~\ref{chapter:implementation} which is directly
followed by measurements and analysis of the results in
Chapter~\ref{chapter:evaluation}. The concluding
Chapter~\ref{chapter:conclusions} summarizes the achieved work and benchmark
results and points out directions for the future.
