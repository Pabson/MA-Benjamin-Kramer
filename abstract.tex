\begin{abstract}
In recent years, hardware characteristics, software systems, and application
patterns have all changed in a way that emphasizes the importance of scan tasks
in modern main-memory database systems. Main memory is available cheaply and
can hold significant amounts of data. In-memory computing has matured and has
resulted in key business platforms such as SAP HANA. Expectations for ad hoc
querying defeat the use of pre-built index structures and mandate scans instead.

At the same time, it is still unclear how the latest advances on the hardware
side can best be leveraged to support efficient scans. Growing SIMD widths, for
instance, appeal with theoretical speed-ups of up $16\times$ or more. But
existing strategies for predicate evaluation and scans are not prepared for such
large SIMD widths. In fact, even for smaller SIMD sizes it is not clear how,
e.~g., predicates that involve multiple data types and widths can be realized
most efficiently.

This thesis studies and evaluates methods that use use of modern hardware
technologies to accelerate scans, in particular \simdscan{}, \emph{BitWeaving}
and \bs{}. Starting from \simdscan{} methods to further reduce memory bandwidth
are explored by skipping parts of the input efficiently and by reducing the
output size by handling intermediate results more efficiently.

\vspace{2cm}\noindent

\end{abstract}
