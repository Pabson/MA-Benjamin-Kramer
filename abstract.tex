\begin{abstract}
In recent years, hardware characteristics, software systems, and application
patterns have all changed in a way that emphasizes the importance of scan tasks
in modern main-memory database systems. Main memory is available cheaply and
can hold significant amounts of data nowadays. In-memory computing has matured
and has resulted in key business platforms such as SAP HANA. Expectations for
ad-hoc querying defeat the use of pre-built index structures and mandate scans
instead.

At the same time, it is still unclear how the latest advances on the hardware
side can best be leveraged to support efficient scans. Growing SIMD widths, for
instance, appeal with theoretical speed-ups of up $16\times$ or more. But
existing strategies for predicate evaluation and scans are not prepared for such
large SIMD widths. In fact, even for smaller SIMD sizes it is not clear how,
e.~g., predicates that involve multiple data types and widths can be realized
most efficiently.

This thesis studies and evaluates methods that use use of modern hardware
technologies to accelerate scans, in particular \simdscan{}, \emph{BitWeaving}
and \bs{}. Starting from \simdscan{} methods to further reduce memory bandwidth
are explored by skipping parts of the input efficiently and by reducing the
output size by handling intermediate results more efficiently.

\vspace{2cm}\noindent

In den vergangenen Jahren haben sich Hard- und Software sowie die Anforderungen
an die Software so geändert, dass der Scan~--~das durchsuchen einer Spalte in
einer Datenbank~--~in In-Memory-Datenbanken immer wichtiger wird. Hauptspeicher
ist günstig und kann heute große Mengen an Daten halten. In-Memory-Datenbanken
sind verbreitet und ausgereift und werden an Schlüsselstellen in der Wirtschaft
verwendet, wie etwa SAP HANA. Die Anforderungen an diese Datenbanken können oft
nicht durch vorberechnete Indices erfüllt werden und benötigen anstelle dessen
Scans.

Gleichwohl ist es immer noch unklar wie man die neusten Errungenschaften in der
Hardware am besten für diese Scans nutzen kann. Breitere SIMD-Register zum
Beispiel, könnten theoretisch eine Beschleunigung um das 16-fache erreichen.
Aber bestehende Scan-Methoden sind nicht auf solch breite Register vorbereitet.
Selbst für kleinere SIMD-Breiten ist es nicht klar wie Prädikate mit mehreren
Datentypen am effizientesten realisiert werden können.

Diese Masterarbeit beschäftigt sich mit Methoden, die diese moderne Hardware
ausnutzen um Scans zu Beschleunigung, mit einem Fokus auf \simdscan{},
\emph{BitWeaving} und \bs{}. Vom Startpunkt des \simdscan{} werden weitere
Möglichkeiten untersucht um die benötigte Speicherbandbreite zu reduzieren, zum
Beispiel durch Überspringen von Teilen der Eingabe oder durch Reduzierung der
Ausgabemenge durch effizientere Verwaltung von Zwischenergebnissen.

\end{abstract}
